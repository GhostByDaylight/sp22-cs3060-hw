\documentclass[paper=letter, fontsize=11pt]{scrartcl} % A4 paper and 11pt font size

\usepackage{enumitem}
\usepackage{listings,multicol}
\usepackage[T1]{fontenc} % Use 8-bit encoding that has 256 glyphs
\usepackage{fourier} % Use the Adobe Utopia font for the document - comment this line to return to the LaTeX default
\usepackage[english]{babel} % English language/hyphenation
\usepackage{amsmath,amsfonts,amsthm} % Math packages
\usepackage{lipsum} % Used for inserting dummy 'Lorem ipsum' text into the template
\usepackage{sectsty} % Allows customizing section commands
\allsectionsfont{\centering \normalfont\scshape} % Make all sections centered, the default font and small caps
\usepackage{fancyhdr} % Custom headers and footers
\pagestyle{fancyplain} % Makes all pages in the document conform to the custom headers and footers
\fancyhead{} % No page header - if you want one, create it in the same way as the footers below
\fancyfoot[L]{} % Empty left footer
\fancyfoot[C]{} % Empty center footer
% \fancyfoot[R]{\thepage} % Page numbering for right footer
\renewcommand{\headrulewidth}{0pt} % Remove header underlines
\renewcommand{\footrulewidth}{0pt} % Remove footer underlines
\setlength{\headheight}{13.6pt} % Customize the height of the header

\setlength\parindent{0pt} % Removes all indentation from paragraphs - comment this line for an assignment with lots of text

\usepackage[margin=0.75in]{geometry}
\usepackage{hyperref}
%----------------------------------------------------------------------------------------
%   TITLE SECTION
%----------------------------------------------------------------------------------------

\newcommand{\horrule}[1]{\rule{\linewidth}{#1}} % Create horizontal rule command with 1 argument of height

\title{ 
    \normalfont \normalsize 
    \textsc{CS 3060 Programming Languages, Spring 2022} \\ [25pt] % Your university, school and/or department name(s)
    \horrule{0.5pt} \\[0.4cm] % Thin top horizontal rule
    \huge Assignment \#1  \\ % The assignment title
    \horrule{2pt} \\[0.5cm] % Thick bottom horizontal rule
}

% \author{John Smith} % Your name

% \date{\normalsize\today} % Today's date or a custom date

\begin{document}

    \begin{center}
         Homework \#1\\
        \small CS 3060 Programming Languages, Spring 2022 \\
        \small Instructor: S. Roy \\
        \huge Ruby \#1
    \end{center}
    
    \textbf{Due Date:}  Jan 25 (11:59 pm).\\

    \textbf{Total points:} 60 points \\

    \textbf{Directions:} For each of the following tasks you need to write a program. 
Then, you run the program and submit the code for each task. You need also submit a README file 
which lists sample input and sample output of all the programs. 
Check Gitlab \@ \texttt{https://gitlab.com/sanroy/sp22-cs3060-hw} (or check Canvas) for details. 
The prefered process for completing this assignment should be as follows:

    \begin{enumerate}[noitemsep]
        \item Fork the Repository ``sp22-cs3060-hw'' to a new Repository named ``sp22-cs3060-hw'' 
under your namespace (your gitlab username).
        \item "git clone" the newly created repository on your local machine
        \item Complete this assignment whose details are in hw1 sub-folder, committing changes to files in hw1. 
        \item Push all commits to your Gitlab repository
        \item Add TA (with his gitlab username kvruakhil2) and Roy (gitlab id is sanroy) as a developer of your Gitlab repository
    \end{enumerate}

If for some reason Gitlab does not work for you, then you submbit the ruby code and README file on Canvas.

    \textbf{Tasks:}
    \begin{enumerate}[noitemsep]

        \item \textbf{(8 Points) program \#1:} Ask the user to type 3 lines 
(e.g., before going to the next line the user will hit the 'Enter' key, etc.) 
on keyboard, and your program should save the lines to a file named "myFile.txt". 
Then, your program also needs to open the file, and for each line report the total number of characters that are present.
        \item \textbf{(6 Points) program \#2:} Ask the user to type the name of a file. 
Then, search the content of the file. 
If the file (content) contains 
``Python'' or ``Ruby'', then print ``The file is important'', else if the file contains ``Scala'', 
then print ``The file is interesting'', else if the file contains ``Haskell'' then print ``the file is valuable''; 
otherwise, print "The file is worthless". 

        \item \textbf{(8 Points) program \#3:} Write a function bar(n) to compute the value of $x^x$ without 
using the exponent operator, (i.e., your need to use the loop syntax) whereas $x$ is a positive integer. 
Test your program at least for 3 values of x such as 4, 7, and 9.

        \item \textbf{(6 Points) program \#4:} Print the string ``The $9$-th multiple of integer $n$ is $x$'' 
while substituting $n$ by numbers from 5 to 13 and $x$ by the value of $9n$.
        \item \textbf{(10 Points) program \#5:} Let the user pick a number (say $x$) between 15 and 30. 
Now your program simulates tossing a coin $x$ times. 
In particular, your program can contain a loop and in each iteration it randomly makes a choice: 
head (represented by string 'HEAD') or 
tail (represented by string 'TAIL'), and stores the outcome ('HEAD' or 'TAIL') 
in an array. After the iterations, traverse the array and count how many heads and tails were generated. 
Also, report the ratio of number of heads and number of tails.

        \item \textbf{(14 Points) program \#6:} Go to \texttt{http://www.textfiles.com/stories/} and check that this site 
\footnote {Disclaimer: we did not really check whether this website contains any improper story or language. 
If you find something improper, please ignore this site and use some other source} hosts multiple stories 
while each story is in a textfile. Download two textfiles of your choice, which have atleast 400 words, 
and save the files as \texttt{storyA.txt} and \texttt{storyB.txt}. Your program needs to read these files and processes 
them to collect some statistics. In particular, for each story $x$ report the the number of unique words in $x$, 
the third-most frequent word in $x$ and its frequency. Also, your program needs to report the number of unique words that are present in both the files.
\textbf {Hints:} A template code is given to you. We use Array and Hash data structures as they are available in Ruby. You may design a regular expression to define a \emph{word}.  

         \item \textbf{(8 Points) program \#7:} Write a C program that prints "Hello, class of CS 3060". 
Use gcc command to compile the program to generate executable file a.out. Submit a screenshot showing the gcc command. 
Then, open the file a.out in an editor and check the content of a.out. Is it readable? Submit a screenshot. 
Then, run a.out and report the outcome. Submit a screenshot. Also, submit your C program.
Now write  a Ruby program that prints "Hello, class of CS 3060". Do you need the compilation step? Also, submit your Ruby program.  
    
\end{enumerate}
 
\end{document}
